\documentclass{article}

\usepackage{verbatim}
\usepackage{tikz}


\title{NexFUSE}

\author{NexFUSE}

\begin{document}

\maketitle
\pagebreak
\tableofcontents
\pagebreak

\section{Introduction}

NexFUSE(c) is a bytecode interpreter which is the successor of the OpenLUD
Bytecode Intermediate format. These programs are meant to try and create a new
bytecode format that is easily accessible and SAFE. This handbook is the guide
to understanding NexFUSE and its commands and while this guide does not go over
practical use cases, it does give examples inside of the descriptions of the
commands on how they can be applied to your everyday programming. Especially if
you work in computer or embedded device development, since NexFUSE contains
little to no memory footprint and only allocates incremental memory, especially
for lists and structures of similar nature.

NexFUSE follows an extremely, if not exact standard to OpenLUD, and is 100\%
organic and backwards compatible with OpenLUD.

\subsection{OpenLUD \& NexFUSE Speed Comparison}

This is a comparison between OpenLUD and NexFUSE. OpenLUD is the bytecode
intermediate format which NexFUSE is based on. NexFUSE also follows similar
rules, however, these rules are used and expanded which makes NexFUSE bytecode
incompatible with OpenLUD. But OpenLUD is able to create bytecode that is
compatible with NexFUSE.

This is a simple speed and memory comparison between OpenLUD and NexFUSE. Do
note that the same exact bytecode was run on both bytecode formats.

\pagebreak

\begin{verbatim}
  OpenLUD:

    Executed in    7.38 millis    fish           external
      usr time    4.01 millis  574.00 micros    3.44 millis
      sys time    3.44 millis    0.00 micros    3.44 millis

    valgrind (...):
      ==31954== HEAP SUMMARY:
      ==31954==     in use at exit: 124 bytes in 9 blocks
      ==31954==   total heap usage: 1,163 allocs, 1,154 frees, 606,235 bytes allocated

  NexFUSE:

    Executed in    2.61 millis    fish           external
      usr time    0.34 millis  339.00 micros    0.00 millis
      sys time    2.38 millis  214.00 micros    2.17 millis

    valgrind (...):
      ==32051== HEAP SUMMARY:
      ==32051==     in use at exit: 0 bytes in 0 blocks
      ==32051==   total heap usage: 67 allocs, 67 frees, 6,605 bytes allocated
      ==32051== 
      ==32051== All heap blocks were freed -- no leaks are possible
\end{verbatim}

Using a simple `valgrind' command, it shows that NexFUSEs memory footprint is
around 25 times smaller than that of OpenLUD. Other `time' outputs also show
that NexFUSE runs 2 times faster than OpenLUD--but \emph{HOW} you may ask?
Because instead of using another programming language (D)'s garbage collector,
NexFUSE instead manages its own memory, while also using an incremental-style
allocation system for lists and objects.

\subsection{NexFUSE Additional Features}

NexFUSE contains an additional feature called \textbf{SUB} that is not a part of
OpenLUD. The \textbf{SUB} command is used to define a subroutine. The following
are the descriptions of the commands:

\begin{itemize}
  \item \textbf{SUB} - defines the start of a subroutine (similar to JMP instructions)
  \item \textbf{ENDSUB} - defines the end of a subroutine
  \item \textbf{GOSUB} - jumps to a subroutine from anywhere in the program
\end{itemize}

\section{NexFUSE Reference}

\subsection{NexFUSE Base (OpenLUD Compatible)}

\subsubsection{\textbf{NULL (`NNULL')}}

The \textbf{NULL} command is used to end a statement call. All statements in
NexFUSE AND OpenLUD are terminated with a \textbf{NULL} byte.

In all distributions of NexFUSE, \textbf{NULL} is defined as \emph{`00'}.

Example:

\begin{verbatim}
  NULL
\end{verbatim}

\subsubsection{\textbf{ECHO}}

The \textbf{ECHO} command is used to print out a byte as a character. NOTE that
this does NOT print out a newline character after it prints the character. Since
characters take up one byte, it will always print the byte as a character, there
is currently no way to print out a byte as a value itself.

Example:
\begin{verbatim}
  ECHO 0x41 NULL # prints `A'
\end{verbatim}

\subsubsection{\textbf{MOVE}}

The \textbf{MOVE} command is used to move a byte into a register. NOTE that this
updates the register pointer, and this is the only function to do so.

Example:
\begin{verbatim}
  MOVE 1 0x41 NULL # moves `A' into register 1
\end{verbatim}

\subsubsection{\textbf{EACH}}

The \textbf{EACH} command loops through a register and prints each byte out if it is
not \textbf{NULL/Unoccupied} (0x00).

Example:
\begin{verbatim}
  MOVE 1 0x41 NULL  // move `A' into register 1
  MOVE 2 0x42 NULL  // move `B' into register 2
  MOVE 3 0x43 NULL  // move `C' into register 3
  MOVE 4 0x0a NULL  // move `\n' into register 4
  EACH 1 NULL       // prints `ABC\n'
\end{verbatim}

\pagebreak
\subsubsection{\textbf{PUT}}

The \textbf{PUT} command writes a byte to a register at a specified position. Do
NOTE that the register pointer is not updated.

Example:
\begin{verbatim}
  PUT 1 0x41 2 NULL # writes `A' into register 1 at position 2
  GET 1 2 2 NULL     # gets Register 1 at position 2 and puts it into register 2
  EACH 2 NULL        # prints `A'
\end{verbatim}

\subsubsection{\textbf{GET}}

The \textbf{GET} command gets a byte from a register at a specified position and
puts it into another register.

Example:
\begin{verbatim}
  MOVE 1 0x41 NULL  // move `A' into register 1
  MOVE 2 0x42 NULL  // move `B' into register 2
  MOVE 3 0x43 NULL  // move `C' into register 3
  MOVE 4 0x0a NULL  // move `\n' into register 4
  GET 1 2 3 NULL    // gets Register 1 at position 2 and puts it into register 3
  EACH 3 NULL       // prints `B'
\end{verbatim}

\subsubsection{\textbf{END}}

The \textbf{END} command is used to end a bytecode. This is used to prevent
infinite loops in NexFUSE. This functionality is different than OpenLUD but END
is still required in both formats.

\subsubsection{\textbf{RESET}}

The \textbf{RESET} command resets the register pointer and its values to 0.

Example:
\begin{verbatim}
  RESET 1 NULL
\end{verbatim}

\subsubsection{\textbf{CLEAR}}

The \textbf{CLEAR} command clears all registers.

Example:
\begin{verbatim}
  CLEAR NULL
\end{verbatim}

\subsection{NexFUSE Additional Features}

These features are not a part of the OpenLUD bytecode format. But added in
NexFUSE, they do not cause any issues with memory or NexFUSEs memory footprint.

\subsubsection{\textbf{SUB}}

The \textbf{SUB} command defines the start of a subroutine. Subroutines must
have an \textbf{END} as well as an \textbf{ENDSUB} command after.

Example:
\begin{verbatim}
  SUB 8
    MOVE 1 0x41 NULL
    END
  ENDSUB
\end{verbatim}

\subsubsection{\textbf{ENDSUB}}

The \textbf{ENDSUB} command defines the end of a subroutine.

Example:
\begin{verbatim}
  SUB 8
    MOVE 1 0x41 NULL
    END
  ENDSUB
\end{verbatim}

\end{document}
